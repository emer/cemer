\documentclass[10pt,landscape]{article}
\usepackage{multicol}
\usepackage{graphicx}
\usepackage{calc}
\usepackage{ifthen}
\usepackage{color}
\usepackage[landscape]{geometry}

\ifthenelse{\lengthtest { \paperwidth = 11in}}
	{ \geometry{top=.5in,left=.5in,right=.5in,bottom=.5in} }
	{\ifthenelse{ \lengthtest{ \paperwidth = 297mm}}
		{\geometry{top=1cm,left=1cm,right=1cm,bottom=1cm} }
		{\geometry{top=1cm,left=1cm,right=1cm,bottom=1cm} }
	}

% Turn off header and footer
\pagestyle{empty}
 

% Redefine section commands to use less space
\makeatletter
\renewcommand{\section}{\@startsection{section}{1}{0mm}%
                                {-1ex plus -.5ex minus -.2ex}%
                                {0.5ex plus .2ex}%x
                                {\normalfont\large\bfseries}}
\renewcommand{\subsection}{\@startsection{subsection}{2}{0mm}%
                                {-1explus -.5ex minus -.2ex}%
                                {0.5ex plus .2ex}%
                                {\normalfont\normalsize\bfseries}}
\renewcommand{\subsubsection}{\@startsection{subsubsection}{3}{0mm}%
                                {-1ex plus -.5ex minus -.2ex}%
                                {1ex plus .2ex}%
                                {\normalfont\small\bfseries}}
\makeatother

% Define BibTeX command
\def\BibTeX{{\rm B\kern-.05em{\sc i\kern-.025em b}\kern-.08em
    T\kern-.1667em\lower.7ex\hbox{E}\kern-.125emX}}

% Don't print section numbers
\setcounter{secnumdepth}{0}


\setlength{\parindent}{0pt}
\setlength{\parskip}{0pt plus 0.5ex}


% -----------------------------------------------------------------------

\begin{document}

\raggedright
\footnotesize
\begin{multicols}{3}


% multicol parameters
% These lengths are set only within the two main columns
%\setlength{\columnseprule}{0.25pt}
\setlength{\premulticols}{1pt}
\setlength{\postmulticols}{1pt}
\setlength{\multicolsep}{1pt}
\setlength{\columnsep}{2pt}

\begin{center}
   \includegraphics[width=15px]{emer.png}\huge{\textbf{mergent keyboard reference}} \\
   
\end{center}
\section{Global project}
\begin{tabular}{@{}ll@{}}
\verb!Ctrl+S!    & Save project. \\
\verb!Ctrl+left! & Backwards in navigation history \\
\verb!Ctrl+right! & Forwards in navigation history \\
\verb!F5!  & Refresh GUI. \\
\verb!Tab! & Forward through interface \\
\verb!Shift+Tab! & Backwards through interface \\
\verb!Alt+J! & Move global focus left \\
\verb!Ctrl+J! & Move global focus left \\
\verb!Alt+L! & Move global focus right \\
\verb!Ctrl+L! & Move global focus right \\
\end{tabular}

\section{Help Browser}
\includegraphics[width=153.5px]{helpbrowser.png} \\
\begin{tabular}{@{}ll@{}}
\verb!F1!  & Help Browser. \\
\verb!Ctrl+S!  & Toggle Search/Find focus. \\
\verb!Tab!  & Switch to search list focus. \\
\verb!Ctrl+n!  & Next element. \\
\verb!Ctrl+p!  & Next element. \\
\verb!Page Up!  & Page Up list or web page. \\
\verb!Page Down!  & Page Down list or web page. \\
\end{tabular}

\section{Tree browser and program code}
\newlength{\MyLen}
\settowidth{\MyLen}{\texttt{letterpaper}/\texttt{a4paper} \ }
\begin{tabular}{@{}ll@{}}
\verb!Any 1-3 chars!    & Find as you type \\
\verb!Alt+F!  & Find from selected node. \\
\verb!Ctrl+I!    & New item below cursor. \\
\verb!Ctrl+F!  & Expand this node. \\
\verb!Shift++!  & Expand this node. \\
\verb!Ctrl+B!  & Collapse this node. \\
\verb!Ctrl+-!  & Collapse this node. \\
\verb!Ctrl+spacebar!  & Selection mode. \\
\verb!Ctrl+P!  & (select) Previous element. \\
\verb!Ctrl+N!  & (select) Next element. \\
\verb!Ctrl+D!  & Delete selected item(s). \\
\verb!Delete!  & Delete selected item(s). \\
\verb!Ctrl+C!  & Copy selected element(s). \\
\verb!Alt+W!  & Copy selected element(s). \\
\verb!Ctrl+X!  & Cut selected element(s). \\
\verb!Ctrl+W!  & Cut selected element(s). \\
\verb!Ctrl+V!  & Paste element(s). \\
\verb!Ctrl+Y!  & Paste element(s). \\
\verb!Ctrl+U!  & Page up. \\
\verb!Cmd+V!  & Page down. (mac only) \\
\verb!Ctrl+G!  & Deselect. \\
\end{tabular}

\section{css console and text fields}
\begin{tabular}{@{}ll@{}}
\verb!Ctrl+A!    & Beginning of line. \\
\verb!Ctrl+E! & End of line. \\
\verb!Ctrl+K! & Kill text until end of line. \\
\verb!Ctrl+B!  & Move cursor back one character. \\
\verb!Ctrl+F! & Move cursor forward one character. \\
\verb!Ctrl+right! & Move cursor one word forward. \\
\verb!Ctrl+left! & Move cursor one word backwards. \\
\verb!Ctrl+shift+right! & Highlight one word forward. \\
\verb!Ctrl+shift+left! & Highlight one word backwards. \\
\verb!Ctrl+X! & Cut. \\
\verb!Ctrl+C! & Copy. \\
\verb!Ctrl+V! & Paste.
\end{tabular}

\subsection{Text fields}
\includegraphics[width=91px]{path.png} \\
\begin{tabular}{@{}ll@{}}
\verb!Ctrl+L!    & Text completion lookup.  \\
\verb!Ctrl+I!    & Insert tab. \\
\verb!Ctrl+U!    & Highlight all.
\end{tabular}

\subsection{css console}
\includegraphics[width=108.5px]{console.png} \\
\begin{tabular}{@{}ll@{}}
\verb!Ctrl+L!    & Clear console buffer history. \\
\verb!Ctrl+L!    & Delete highlighted text. 
\end{tabular}


\section{DataTables and matrices}
\includegraphics[width=115px]{datatable.png} \\
\begin{tabular}{@{}ll@{}}
\verb!Ctrl+T!    & Switch between table and matrix focus. \\
\verb!Ctrl+F!    & Move one cell right. \\
\verb!Ctrl+B!    & Move one cell left. \\
\verb!Ctrl+P!    & Move one cell up. \\
\verb!Ctrl+N!    & Move one cell down. \\
\verb!Ctrl+C!    & Copy table/matrix cell. \\
\verb!Ctrl+V!    & paste table/matrix cell. \\
\end{tabular}


\section{New elements in left tree browser}
\includegraphics[width=66px]{treebrowser.png} \\
\begin{tabular}{@{}ll@{}}
\verb!do Ctrl+I!    & New Doc \\
\verb!da Ctrl+I!    & New DataTable \\
\verb!la Ctrl+I!    & New Layer \\
\verb!P Ctrl+I!    & New Project \\
\verb!pr Ctrl+I!    & New Program \\
\verb!n Ctrl+I!    & New Network \\
\verb!sp Ctrl+I!    & New Spec
\end{tabular}

\section{New elements in program code}
\includegraphics[width=56px]{programcode.png} \\
These sequences insert new items. Press Ctrl+left,left afterwards to navigate back to where you were. \\
\begin{tabular}{@{}ll@{}}
\verb!obj Ctrl+I Type!    & New obj of Type \\
\verb!var Ctrl+I!    & New var \\
\verb!arg Ctrl+I!    & New arg \\
\verb!fun Ctrl+I!    & New fun \\
\verb!init Ctrl+I Name!    & New init code Name \\
\verb!prog Ctrl+I Name!    & New prog code Name \\
\end{tabular}

\section{Middle panel edit dialogs}
\includegraphics[width=225px]{middlepanel.png} \\
\settowidth{\MyLen}{\texttt{multicol} }
\begin{tabular}{@{}ll@{}} \\
\verb!Tab!    & Next element. \\
\verb!Ctrl+n!    & Next element. \\
\verb!Shift+tab!    & previous element. \\
\verb!Ctrl+p!    & Next element. \\
\verb!Up!    & (numeric field) Increase value. \\
\verb!Down!    & (numeric field) Decrease value. \\
\verb!Up!    & (dropdown) Move up. \\
\verb!Down!    & (dropdown) Move down. \\
\verb!ESC!    & Revert changes. \\
\verb!Ctrl+Enter!    & Apply changes. \\
\verb!Spacebar!    & (buttons) Open token chooser. \\
\verb!Spacebar!    & (flags) Check/uncheck flag. \\
\verb!Ctrl+L!    & (expression fields) Lookup information. \\
\end{tabular}

\section{Program code elements}
\includegraphics[width=137px]{progel.png} \\
Press Ctrl+i seq Enter as fast as you can, where seq is the shortest subset of the full name needed to put that program element at the top of the chooser list. No need to wait for visual confirmation of the choice.

\end{multicols}
\end{document}
